%%%%%%%%%%%%%%%%%%%%%%%%%%%%%%%%%%%%%%%%%%%%%%%%%%%%%%%%%%%%%%%%%%%%%%
% LaTeX Example: Project Report
%
% Source: http://www.howtotex.com
%
% Feel free to distribute this example, but please keep the referral
% to howtotex.com
% Date: March 2011 
% 
%%%%%%%%%%%%%%%%%%%%%%%%%%%%%%%%%%%%%%%%%%%%%%%%%%%%%%%%%%%%%%%%%%%%%%
% How to use writeLaTeX: 
%
% You edit the source code here on the left, and the preview on the
% right shows you the result within a few seconds.
%
% Bookmark this page and share the URL with your co-authors. They can
% edit at the same time!
%
% You can upload figures, bibliographies, custom classes and
% styles using the files menu.
%
% If you're new to LaTeX, the wikibook is a great place to start:
% http://en.wikibooks.org/wiki/LaTeX
%
%%%%%%%%%%%%%%%%%%%%%%%%%%%%%%%%%%%%%%%%%%%%%%%%%%%%%%%%%%%%%%%%%%%%%%
% Edit the title below to update the display in My Documents
%\title{Project Report}
%
%%% Preamble
\documentclass[paper=a4, fontsize=11pt]{scrartcl}
\usepackage[T1]{fontenc}
\usepackage{xparse}
\usepackage{fourier}
\usepackage{mathtools}
\DeclarePairedDelimiter\ceil{\lceil}{\rceil}
\DeclarePairedDelimiter\floor{\lfloor}{\rfloor}
\usepackage{listings}
\usepackage[english]{babel}                                                         % English language/hyphenation
\usepackage[protrusion=true,expansion=true]{microtype}  
\usepackage{amsmath,amsfonts,amsthm} % Math packages
\usepackage[pdftex]{graphicx}   
\usepackage{url}


%%% Custom sectioning
\usepackage{sectsty}
\allsectionsfont{\centering \normalfont\scshape}


%%% Custom headers/footers (fancyhdr package)
\usepackage{fancyhdr}
\usepackage{listings}
\pagestyle{fancyplain}
\fancyhead{}                                            % No page header
\fancyfoot[L]{}                                         % Empty 
\fancyfoot[C]{}                                         % Empty
\fancyfoot[R]{\thepage}                                 % Pagenumbering
\renewcommand{\headrulewidth}{0pt}          % Remove header underlines
\renewcommand{\footrulewidth}{0pt}              % Remove footer underlines
\setlength{\headheight}{13.6pt}


%%% Equation and float numbering
\numberwithin{equation}{section}        % Equationnumbering: section.eq#
\numberwithin{figure}{section}          % Figurenumbering: section.fig#
\numberwithin{table}{section}               % Tablenumbering: section.tab#


%%% Maketitle metadata
\newcommand{\horrule}[1]{\rule{\linewidth}{#1}}     % Horizontal rule

\title{
        %\vspace{-1in}  
        \usefont{OT1}{bch}{b}{n}
        \normalfont \normalsize \textsc{University of Illinois at Urbana-Champaign} \\ [25pt]
        \horrule{0.5pt} \\[0.4cm]
        \huge Assignment 4 - Report \\
        \horrule{2pt} \\[0.5cm]
}
\author{
        \normalfont                                 \normalsize
        Department of Industrial and Enterprise Systems Engineering\\
        \normalsize Zhenye Na (zna2)\\[-3pt]        \normalsize
        \today
}
\date{}


%%% Begin document
\begin{document}
\maketitle

\section{Indexing (10 pts)}

\begin{enumerate}
    \item Suppose we want to build an index on a relation R which has a total of x records, with each block capable of holding either y records or z key-pointer pairs. Assuming x is divisible by y, please answer the following questions (if your value evaluates to a fraction, use ceiling $\floor*{\;\;}$ or floor $\ceil*{\;\;}$ as appropriate):\\


\begin{enumerate}
    \item Suppose you construct a simple single level index, and that index is dense. How many index blocks are required to access all of the records of R?\\
    \textbf{Solutions: }


    \item Suppose the index built is sparse. If the index stores a pointer to the lowest search key in each block, and the index is a simple single level index, how many data blocks do we need? How many index blocks do we need?\\
    \textbf{Solutions: }

\end{enumerate}



    \item True/False question - In order to use a dense index, you will have to have the data file sorted by the search key; otherwise, you will need to use a sparse index. Explain your reasoning.\\
    \textbf{Solutions: }

        
\end{enumerate}


\section{B+ tree (30 pts)}

\[
\includegraphics[scale=0.4]{1.png}
\]

\begin{enumerate}
    \item Draw the B+ tree that would result from inserting a data entry with key 13.\\
    \textbf{Solutions: }


    \item Based on the B+ tree that you drew in the previous question, draw the B+ tree that would result from deleting the data entry with key 75.\\
    \textbf{Solutions: }


    \item Based on the B+ tree that you drew in the previous question, draw the B+ tree that would result from deleting the data entry with key 89.\\
    \textbf{Solutions: }


\end{enumerate}


\section{Extensible Hashing (30 pts)}

Assume you have a extensible hash table with hash function h(k) = k mod 13, expressed as a binary string of size 4, and data block of size 2 (i.e., it can accommodate two tuples). You are asked to index the following key values in order: 25, 13, 23, 21.

\begin{enumerate}
    \item Draw the extensible hash table which obeys the above constraints after the four keys are inserted.\\
    \textbf{Solutions: }


    \item Using your solution to the previous question, now consider insertion of keys 18 and 20 into the hash table, and draw the resulting hash table.\\
    \textbf{Solutions: }


\end{enumerate}



\section{Linear Hashing (30 pts)}

Consider a linear hash table with $r \leq 1.76n$ with each data block capable of holding 2 records (that is, the average number of record per bucket should not exceed $88\%$ of the total number of records per block):

\[
\includegraphics[scale=0.4]{2.png}
\]

\begin{enumerate}
    \item Insert 1001 and draw the resulting table.\\
    \textbf{Solutions: }

    \item With your solution from the previous question, insert 1101, 1110, 0001 incrementally and draw the final table; that is, insert one at a time, check the condition, and move to the next one.\\
    \textbf{Solutions: }


\end{enumerate}





%%% End document
\end{document}